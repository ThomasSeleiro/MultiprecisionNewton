\documentclass[10pt, A4paper]{article}

\usepackage{amsmath}
\usepackage{amssymb}
\usepackage{amsthm}
\usepackage{mathtools}
\usepackage[linesnumbered,ruled]{algorithm2e}
\usepackage{booktabs}
\usepackage{graphicx}
\usepackage{subcaption}

\usepackage{fancyvrb}

\usepackage{natbib}
\setcitestyle{numbers, square}

\newtheorem{theorem}{Theorem}[section]
\newtheorem{lemma}[theorem]{Lemma}
\newtheorem{corollary}[theorem]{Corollary}

\newcommand{\mxm}{m \times m}
\newcommand{\mxn}{m \times n}
\newcommand{\nxn}{n \times n}
\DeclareMathOperator{\diag}{diag}
\DeclareMathOperator{\rank}{rank}
\DeclareMathOperator{\sign}{sign}
\DeclareMathOperator{\matVec}{vec}
\DeclareMathOperator{\trace}{trace}

\newcommand*{\consoleFont}{\fontfamily{pcr}\selectfont}

%%%%%%%%%%%%%%%%%%%%%%%%%%%%%%%%%%%%%%%%%%%%%%%%%%%%%%%%%%%%%%%%%%%%%%

\begin{document}

\title{Note on Trace Maximization Correction to the Multi-precision
Polar Decomposition}
\author{Thomas Seleiro\thanks
	{Department of Mathematics, University of Manchester,
	Manchester, M13 9PL, UK
	(\texttt{thomas.seleiro@postgrad.manchester.ac.uk})}}
\date{11 January 2021}
\maketitle


\section{Polar Decomposition}

For $A \in \mathbb{C}^{\mxn}$, we can find a polar decomposition $A =
UH$, where $U \in \mathbb{C}^{\mxn}$ has unitary columns and $H \in
\mathbb{C}^{\nxn}$ is Hermitian positive semi-definite.
The unitary polar factor $U$ for an $\nxn$ matrix $A$ can be computed
via the scaled Newton iteration defined by the recursive step
$$X_{k+1} = \dfrac{1}{2} (\mu_k X_k + \mu_k^{-1} X_k^{-*}),
\qquad X_0 = A.$$
Throughout the experiments performed, we used the $1,\infty$-norm
scaling factor
$$\mu_k = \left( \dfrac{\|X_k^{-1}\|_1 \|X_k^{-1}\|_\infty}
{\|X_k\|_1 \|X_k\|_\infty}\right)^{1/4},$$
and we use a mixture of the stopping conditions
$\|X_k - X_{k-1}\|_\infty / \|X_k\|_\infty \leq nu$
and
$\|I - X_k^*X_k\|_\infty \leq nu$ suggested in~{[\citealp{high2008},
\S8.4].}

We try to evaluate the effectiveness of using this method for computing
the polar decomposition of a matrix in multiple precision.
The iterates converge quadratically to the unitary polar factor.
Therefore once the iteration has converged to a lower precision, only
one further iteration in the desired higher precision would be needed
for convergence.

The computed matrix $U_1$ will be unitary to the desired precision, but
the corresponding Hermitian factor $H_1 = U_1^*A$ need not be Hermitian
positive semi-definite.
\begin{table}
	\centering
	\begin{tabular}{cccc}
		\toprule

		\bottomrule
	\end{tabular}
	\caption{\label{}
		}
\end{table}
Table~\ref{tab:}. shows that $H_1$ is only Hermitian to single
precision and the calculated matrices are inaccurate.
We try to compensate for this inaccuracy by calculating the polar
decomposition $H_1 = WH$. We then have $A = U_1H_1 = (U_1W)H \eqqcolon
UH$, where $U$ is unitary to double precision and $H$ is Hermitian
positive semi-definite.

In general, $H_1$ is not unitary ($\|A\|_2 = \|U_1H_1\|_2 =
\|H_1\|_2$). Therefore we cannot use the Newton method to compute this
polar decomposition, since the iterates converge to a unitary matrix.

We instead use the property that for all unitary
$W \in \mathbb{C}^{\nxn}$, $\trace(W^*A)$ is maximised if and only if
$W$ is a unitary polar factor of $A$~{(see [\citealp{high2008}, Prob.
8.13])}.

An algorithm for computing the polar decomposition is proposed
in~{[\citealp{smit2002}, p.84]}.
We repeatedly loop through every $2\times 2$ submatrix $A_{ij} =
A([i,j], [i,j])$ and apply Givens transformations that make $A_{ij}$
symmetric and maximise its trace.
We do so until the matrix is symmetric.
If the resulting matrix $\tilde{A}$ is indefinite, we apply a
Householder transformation $W^*$ where $W = I - 2xx^*$ where $x$ is the
normalized eigenvector of the smallest eigenvalue of $\tilde{A}$.
This makes the resulting matrix symmetric positive semi-definite and
increases the trace of $\tilde{A}$.

\begin{table}
	\centering
	\begin{tabular}{lccccc}
		\toprule
		& $t_G$ &	$t_N$&	$s_G$&	$\|H_G - H_G^*\|/2$ &
		$\|H_N - H_N^*\|/2$ \\
		\midrule
		\texttt{rand(25)}&	0.27	&1.67e-3&	245&	1.91e-14&
		8.12e-15\\
		\texttt{rand(50)}&	4.88&	1.83e-3&	713&	7.88e-14&
		3.90e-14	\\
		\texttt{rand(75)}&	35.96&	3.11e-3&	1364&	1.64e-13&
		9.17e-14\\
		\texttt{rand(100)}&	140.53&	4.49e-3	&2110&	3.14e-13&	1.46e-13\\
		\bottomrule
	\end{tabular}
	\caption{\label{tab:mtvsdouble}
	Table showing $t_G$ and $t_N$, the calculation times using 
	\texttt{maxtracePoldec} and a double precision Newton iteration; 
	$s_G$ the number of sweeps of \texttt{maxtracePoldec}, and the norm 
	skew-Hermitian parts of the computed Hermitian polar factors.
	}
\end{table}


\bibliography{MultiprecisionBib}
\bibliographystyle{siam}

\end{document}
