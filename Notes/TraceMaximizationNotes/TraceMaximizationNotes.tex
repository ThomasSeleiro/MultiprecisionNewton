\documentclass[10pt, A4paper]{article}

\usepackage{amsmath}
\usepackage{amssymb}
\usepackage{amsthm}
\usepackage{mathtools}
\usepackage[linesnumbered,ruled]{algorithm2e}
\usepackage{booktabs}
\usepackage{graphicx}
\usepackage{subcaption}

\usepackage{fancyvrb}
\usepackage{adjustbox}

\usepackage{natbib}
\setcitestyle{numbers, square}

\newtheorem{theorem}{Theorem}[section]
\newtheorem{lemma}[theorem]{Lemma}
\newtheorem{corollary}[theorem]{Corollary}

\newcommand{\mxm}{m \times m}
\newcommand{\mxn}{m \times n}
\newcommand{\nxn}{n \times n}
\DeclareMathOperator{\diag}{diag}
\DeclareMathOperator{\rank}{rank}
\DeclareMathOperator{\sign}{sign}
\DeclareMathOperator{\matVec}{vec}
\DeclareMathOperator{\trace}{trace}

\newcommand*{\consoleFont}{\fontfamily{pcr}\selectfont}

%%%%%%%%%%%%%%%%%%%%%%%%%%%%%%%%%%%%%%%%%%%%%%%%%%%%%%%%%%%%%%%%%%%%%%

\begin{document}

\title{Note on Trace Maximization Correction to the Multi-precision
Polar Decomposition}
\author{Thomas Seleiro\thanks
	{Department of Mathematics, University of Manchester,
	Manchester, M13 9PL, UK
	(\texttt{thomas.seleiro@postgrad.manchester.ac.uk})}}
\date{11 January 2021}
\maketitle


\section{Polar Decomposition}

For $A \in \mathbb{C}^{\mxn}$ with $m \geq n$, we can find a polar 
decomposition $A =
UH$, where $U \in \mathbb{C}^{\mxn}$ has orthonormal columns and $H \in
\mathbb{C}^{\nxn}$ is Hermitian positive semi-definite.
The unitary polar factor $U$ for a non-singular $\nxn$ matrix $A$ can 
be computed
via the scaled Newton iteration defined by the recursive step
$$X_{k+1} = \dfrac{1}{2} (\mu_k X_k + \mu_k^{-1} X_k^{-*}),
\qquad X_0 = A.$$
Throughout the experiments performed, we used the $1,\infty$-norm
scaling factor
$$\mu_k = \left( \dfrac{\|X_k^{-1}\|_1 \|X_k^{-1}\|_\infty}
{\|X_k\|_1 \|X_k\|_\infty}\right)^{1/4},$$
and we used a mixture of the stopping conditions
$\|X_k - X_{k-1}\|_\infty / \|X_k\|_\infty \leq nu$
and
$\|I - X_k^*X_k\|_\infty \leq nu$ suggested in~{[\citealp{high2008},
\S8.4].}

We try to evaluate the effectiveness of using this method for computing
the polar decomposition of a matrix in multiple precision.
The iterates converge quadratically to the unitary polar factor.
Therefore once the iteration has converged to a lower precision, only
one further iteration in the desired higher precision would be needed
for convergence.

The computed matrix $U_1$ will be unitary to the desired precision, but
the corresponding Hermitian factor $H_1 = U_1^*A$ need not be Hermitian
positive semi-definite.
\begin{table}
	\centering
	\begin{tabular}{lccc}
		\toprule
		$A$   & $\|H_{mp} - H^*_{mp}\|_\infty / 2$ & $\|H - H^*\|_\infty / 2$
		& $\|U - U_{mp}\|_\infty$ \\
		\midrule
		\texttt{rand(20)}  & 2.52e-06         & 4.72e-15         & 1.14e-05 \\
		\texttt{rand(40)}  & 1.09e-05         & 2.35e-14         & 3.35e-05 \\
		\texttt{rand(60)}  & 2.84e-05         & 4.92e-14         & 6.56e-05 \\
		\texttt{rand(80)}  & 4.07e-05         & 8.39e-14         & 9.33e-05 \\
		\texttt{rand(100)} & 6.84e-05         & 1.42e-13         & 1.38e-04 \\
		\bottomrule
	\end{tabular}
	\caption{
		\label{tab:multiPoldec}
		Table comparing results between the multi-precision polar 
		decomposition Newton iteration (without correcting the 
		resulting Hermitian polar factor), and the corresponding 
		iteration in double precision. Note ``mp'' refers to 
		multi-precision.
	}
\end{table}
Table~\ref{tab:multiPoldec}. shows that in multi-precision $H_1$ is 
only Hermitian to 
single precision and the calculated matrices are inaccurate.
We try to compensate for this inaccuracy by calculating the polar
decomposition $H_1 = WH$. We then have $A = U_1H_1 = (U_1W)H \eqqcolon
UH$, where $U$ is unitary to double precision and $H$ is Hermitian
positive semi-definite.

In general, $H_1$ is not unitary ($\|A\|_2 = \|U_1H_1\|_2 =
\|H_1\|_2$). Therefore we try to avoid using the Newton method to
compute this polar decomposition, since the iterates converge to a
unitary matrix.

We instead consider the property that for all unitary
$W \in \mathbb{C}^{\nxn}$, $\trace(W^*A)$ is maximised if and only if
$W$ is a unitary polar factor of $A$~{(see [\citealp{high2008}, Prob.
8.13])}.

An algorithm for computing the polar decomposition using the trace 
maximisation property is proposed in~{[\citealp{smit2002}, p.84]}.
We repeatedly loop through every $2\times 2$ principal submatrix 
$A_{ij} =A([i,j], [i,j])$ and apply Givens transformations that make 
$A_{ij}$ symmetric and maximise its trace.
We do so until the matrix is symmetric.
If the resulting matrix $\tilde{A}$ is indefinite, it has a smallest 
negative eigenvalue $\lambda_{min}(A)$ and associated eigenvector $x$. 
Applying a Householder transformation $W^*$ where $W = I - 2xx^*$ makes 
the resulting matrix Hermitian positive semi-definite and
increases the trace of $\tilde{A}$.

We directly implemented this algorithm in the function
\texttt{maxtracePoldec}.
Our implementation differs from~[\citealp{smit2002}] by adding a
relaxation term to the symmetric condition.
\begin{Verbatim}
symmDist = norm(A - A', inf) / norm(A, inf);
while(symmDist > u*n)
\end{Verbatim}
We added this term since for random dense matrices of moderate size,
the routine remains stuck in the while loop.

As a method for computing a general polar decomposition,
Table~\ref{tab:mtvsdouble}. shows that the Newton method is
more efficient and accurate than the trace maximization algorithm.
\begin{table}
	\centering
	\begin{tabular}{lccccc}
		\toprule
		$A$ & $t_G$ &	$t_N$&	$s_G$&	$\|H_G - H_G^*\|/2$ &
		$\|H_N - H_N^*\|/2$ \\
		\midrule
		\texttt{rand(25)}&	0.27	&1.67e-3&	245&	1.91e-14&
		8.12e-15\\
		\texttt{rand(50)}&	4.88&	1.83e-3&	713&	7.88e-14&
		3.90e-14	\\
		\texttt{rand(75)}&	35.96&	3.11e-3&	1364&	1.64e-13&
		9.17e-14\\
		\texttt{rand(100)}&	140.53&	4.49e-3	&2110&	3.14e-13&
		1.46e-13\\
		\bottomrule
	\end{tabular}
	\caption{\label{tab:mtvsdouble}
		Table showing $t_G$ and $t_N$, the calculation times using
		\texttt{maxtracePoldec} and a double precision Newton iteration;
		$s_G$ the number of sweeps of \texttt{maxtracePoldec}, and the
		norm
		skew-Hermitian parts of the computed Hermitian polar factors.
	}
\end{table}
Looking at the output of \texttt{maxtracePoldec} in
Fig.~\ref{fig:maxtraceOutput}, we see that the convergence of the trace
maximization algorithm is linear and thus unusable to efficiently
correct the multi-precision result.
One observation that can be made when looking at the output is that the
trace varies very little after relatively few sweeps of the algorithm.
This could leave space for using a two step method, one which rapidly
maximises the trace, and another which focusses on making the matrix
symmetric rapidly.
Such a method could allow for the partial use of this algorithm, over a
restricted number of sweeps to maximise the trace of $H_1$.
\begin{figure}
	\centering
{\small
\begin{BVerbatim}
Sweep     |A-A'|/|A|     trace_Diff
===================================
1         1.9541e-07     6.2148e-14
2         1.3377e-07     1.0236e-14
3         8.0454e-08     2.4372e-15
4         5.7389e-08     1.5842e-15
5         3.9764e-08     9.7488e-16
6         3.1497e-08     8.5302e-16
7         2.5357e-08     4.8744e-16
8         2.2810e-08     1.2186e-16
...
453       3.8064e-15     0.0000e+00
454       3.6931e-15     0.0000e+00
455       3.5531e-15     1.2186e-16
456       3.4020e-15     1.2186e-16
\end{BVerbatim}
}
	\caption{\label{fig:maxtraceOutput}
		Output of \texttt{maxtracePoldec} when used to correct the
		multi-precision polar decomposition of \texttt{A = rand(32)}
	}
\end{figure}

A formula for the polar decomposition of a matrix $B \in \mathbb{R}^   
{2\times 2}$ is known~{[\citealp{high2008}, Prob. 8.2]}.
A variant of the algorithm considered involves directly computing the
polar decomposition of the $2\times 2$ submatrices $A_{ij}$.
The polar decomposition maximises the trace (over all unitary 
matrices), thus we could not find any better unitary matrices to use in 
the algorithm.
Note that such a method still requires a final check for positive
semi-definiteness and potentially a Householder correction like in
\texttt{maxtracePoldec}.

We implemented this variant of the algorithm in the function
\texttt{twobytwoPoldec}.
We then compared its performance computing the multi-precision
correction against that of the Givens rotation based algorithm, and a
simple Newton iteration on $H_1$.
As Table~\ref{tab:GivPolNewComp}. shows, there is no significant
difference in runtime or accuracy of the computed correction with
either trace maximisation method.
More importantly, these methods fail in both these metrics when
compared to calculating the correction with a Newton method.
\begin{table}[t]
    \centering
    \begin{adjustbox}{center}
    \begin{tabular}{lcccccccc}
		\toprule
	 & \multicolumn{3}{c}{Runtime} & \multicolumn{2}{c}{Sweeps}&
	\multicolumn{3}{c}{$\|H - H^*\|_\infty / 2$}\\
	\cmidrule(lr){2-4} \cmidrule(lr){5-6} \cmidrule(lr){7-9}
    $A$ & $t_G$ & $t_P$ & $t_N$ & $s_G$ & $s_P$ & $H_G$ &
    $H_P$ & $H_N$ \\
		\midrule
\texttt{rand(20)}  & 0.12      & 0.12      & 9.85e-04  & 176       & 176       & 1.25e-14         & 1.27e-14         & 1.70e-15         \\
\texttt{rand(40)}  & 0.66      & 0.64      & 1.32e-03  & 182       & 183       & 4.92e-14         & 4.91e-14         & 5.06e-15         \\
\texttt{rand(60)}  & 6.41      & 6.69      & 2.07e-03  & 420       & 421       & 1.16e-13         & 1.16e-13         & 8.80e-15         \\
\texttt{rand(80)}  & 16.68     & 16.58     & 3.56e-03  & 446       & 447       & 1.97e-13         & 1.94e-13         & 1.20e-14         \\
\texttt{rand(100)} & 41.50     & 40.30     & 4.65e-03  & 552       & 552       & 3.13e-13         & 3.16e-13         & 1.63e-14         \\
  	\bottomrule
    \end{tabular}
	\end{adjustbox}
    \caption{\label{tab:GivPolNewComp}
    	Table showing the runtime, number of sweeps, and norm of the
    	skew-Hermitian component of the computed correction to the
    	multi-precision Newton method of $A$.
    	The subscripts $G$, $P$ and $N$ correspond respectively to
    	corrections using \texttt{maxtracePoldec},
    	\texttt{twobytwoPoldec} and a double precision Newton method on
		$H_1$
	}
\end{table}
We also note that \texttt{twobytwoPoldec} exhibits similar behaviour to
\texttt{maxtracePoldec} in Fig~\ref{fig:maxtraceOutput}.
However, the trace difference doesn't drop as low as
\texttt{maxtracePoldec} and stays around $10^{-15}$.


\section{Implementation of the methods}

The implementation of a flexible single-double precision Newton 
Iteration for the polar decomposition is contained in 
\emph{multiPoldec.m}:

{\small
	\VerbatimInput{../code/poldec.m}
}

\bibliography{multi-precisionBib}
\bibliographystyle{siam}

\end{document}
